\documentclass{article}
\usepackage{amsfonts}
\usepackage{amsmath}
\begin{document}
	
% % % Page 1 % % %
\textbf{\LARGE Author: Keith Stellyes}\\

\textbf{\Large A few proofs regarding bitwise operations.}\\

Page 2: \texttt{k << 1 = k * 2}

Page 3: \texttt{1 << k = $2^{k}$}

Page 4: \texttt{k \& 1 =  isOdd(k)}

Page 5: \texttt{k >> 1 = k div 2}

\newpage

% % % Page 2 % % %

\textbf{Proof \texttt{k << 1 = k * 2}} where \texttt{<<} is the bitshift left operation and k is an integer represented as a series of bits.\

The bit string ordered set $B$ is equal to the integer $I(B)$ where:

$$I(B) = \sum_{k=0}^{|B| - 1} 2^{k} \times B_{k}$$

For convenience, we define a function, $bi(b, k)$ where $b = 0 \oplus b = 1$ where:

$$bi(b, k) = 2^{k} \times b$$

Which lets us re-express $I(B)$ as:

$$I(B) = \sum_{k=0}^{|B| - 1} bi(B_{k}, k)$$

The bit string representing the left bitshift is defined by the function $lshift(B)$ where:

$$lshift(B)_{k} = \begin{cases}
0 & \mbox{if } k \leq 0\\
B_{k - 1} & \mbox{if } k > 0
\end{cases}
$$

We want to prove the following: $$I(lshift(B)) = 2 \times I(B) = $$ 

$$\sum_{k=0}^{|B| - 1} bi(B_{k}, k) = 2 \times \sum_{k=0}^{|B| - 1} bi(lshift(B)_{k}, k)$$

$$bi(m, n + 1) = 2 \times bi(m, n)$$

$$2 \times 2^{n} \times m = 2^{n + 1} \times m$$

For $k > 0$\\

$k+1$ to maintain the magnitude...:

$$bi(lshift(B)_{k}, k) = bi({B_{k - 1}, k + 1}) = 2 \times bi(B_{k - 1}, k)$$

$$\sum_{k=0}^{|B| - 1} bi(B_{k}, k) = 2 \times \sum_{k=0}^{|B| - 1} bi(B_{k - 1}, k) = $$

$$2 \times \sum_{k=0}^{|B| - 1} bi(leftshift(B_{k}), k)$$

% % % Page 3 % % %
\newpage
\textbf{Proof \texttt{1 << k = $2^{k}$}}\\

\textbf{Case I.} $k = 0$\

$$n << 0 = n \text{ where } n \in \mathbb{Z}$$

$$1 \texttt{<<} 0 = 1$$

\textbf{Case II.} $k > 0$

$$\text{Where } n > 0 \text{ and } n \in \mathbb{Z}$$
$$m \texttt{<<} n = m \times 2_{1} \times 2_{2} \ldots 2_{n - 1}$$
$$2 \texttt{<<} n = 2 \times 2_{1} \times 2_{2} \times 2_{2} \ldots 2_{n - 1}$$
$$2 \texttt{<<} k = 2^{k}$$

% % % Page 4 % % %
\newpage
\textbf{Proof \texttt{k \& 1 =  isOdd(k)}}\\

$A$ \texttt{\&} $B$ is the bitwise AND which returns a set defined as:

$$(A \texttt{ \& } B)_{k} = A_{k} \times B_{k}$$

$$(A \texttt{ \& } B)_{k} = \begin{cases}
1 & \mbox{if } A_{k} = B_{k} = 1\\
0 & \mbox{else}
\end{cases}
$$

Prove:\

$I(A \texttt{ \& } M) = 1 \rightarrow I(A \texttt{ \& } B) = 2k + 1$\

$I(A \texttt{ \& } M) = 0 \rightarrow I(A \texttt{ \& } B) = 2k$\\

Where $M$ is a set where $k = 0 \rightarrow M_{k} = 1$ and $k > 0 \rightarrow M_{k} = 0$

1. The sum of two even numbers is also even. \
$$\text{Where } i,j,k \in \mathbb{Z}: 2i + 2j = 2k $$

2. The sum of an even number and an odd number is odd.
$$\text{Where } i,j,k \in \mathbb{Z}: 2i + 1 + 2j = (2i + 2j) + 1 = 2k + 1$$

3. A power of 2 is even if that power is an integer greater than 0.

$$2^{n} \rightarrow 2^n = 2k \text{ where } n > 0$$

\textbf{Case I. I(A) = 2k + 1}

$$2^{0} = 2k+1$$
$$A_{0} = 1 \rightarrow I(B) = 2k+1$$
$$(A \texttt{ \& } M)_{0} = 1 \rightarrow A_{0} = 1 \rightarrow I(A) = 2k+1$$

\textbf{Case II. I(A) = 2k}

$$A_{0} = 0 \rightarrow I(A) \rightarrow 2k$$
$$(A \texttt{ \& } M)_{0} = 0 \rightarrow A_{0} = 0$$
$$I(A) = 2k$$

% % % Page 5 % % %
\newpage
\textbf{Proof \texttt{n >> 1 = n div 2}}

$$rshift(B)_{k} = \begin{cases}
0 & \mbox{if } k = |B|-1\\
B_{k + 1} & \mbox{if } k < |B| - 1
\end{cases}
$$

Where $k \in \mathbb{Z}$:

Even number = $2k$

Odd number = $2k + 1$

We define the $a$ \texttt{div} $2$ operator as follows:

$$a \texttt{ div } 2 = \begin{cases}
a \div 2 & \mbox{if } a = 2k\\
(a - 1) \div 2 & \mbox{if } a = 2k + 1
\end{cases}
$$

If $B_{0} = 0$:\

$$B = rshift(lshift(B)) = lshift(rshift(B))$$

We must prove the following:

$$I(rshift(B)) = I(B) \texttt{ div } 2$$

To do so, we will prove by cases:\\

\textbf{Case I.} $I(B) = 2k$

$$n \texttt{>>} 1 = n \text{ div } 2$$
$$n \texttt{>>} 1 = \frac{n}{2}$$
$$(\texttt{<<}) n \texttt{>>} 1 = 2 \times \frac{n}{2}$$
$$n = n$$

\textbf{Case II.} $I(B) = 2k + 1$

$$n \texttt{>>} 1 = n \text{ div } 2$$
$$n \texttt{>>} 1 = \frac{n - 1}{2}$$
$$n \texttt{>>} 1 = (n - 1) \texttt{>>} 1$$
$$(n - 1) \texttt{>>} 1 = \frac{n - 1}{2}$$
$$n - 1 = n - 1$$
$$n = n$$

\end{document}